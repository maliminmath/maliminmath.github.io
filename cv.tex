%%%%%%%%%%%%%%%%%%%%%%%%%%%%%%%%%%%%%%%%%%%%%%%%%%%%%%%%%%%%%%%%%%%%%%%%%%%%%%
%                             Curriculum Vitae                               %
%%%%%%%%%%%%%%%%%%%%%%%%%%%%%%%%%%%%%%%%%%%%%%%%%%%%%%%%%%%%%%%%%%%%%%%%%%%%%%
\documentclass[10pt,a4]{article}
\topmargin-2.0cm
\advance\oddsidemargin-1.2cm
\advance\evensidemargin-1.2cm
\textheight9.22in
\textwidth6.4in
\newcommand\bb[1]{\mbox{\em #1}}
%\def\baselinestretch{1.25}
\def\baselinestretch{1.0}

\usepackage{multicol}
% The use of the times package forces the use of the type-1 times
% roman font, but the times roman font does not look nice.
% Besides the times roman font still does not print correctly on
% the dopy printer.
%\usepackage{times}

\usepackage{fancyhdr}
\usepackage{origpagecounting}
\usepackage[dvips]{color}

%\newcounter{myEnumCounter}
%\newcounter{mySaveCounter}
%\renewenvironment{enumerate}{%
%  \begin{list}{\arabic{myEnumCounter}.}{\usecounter{myEnumCounter}%
%  \setcounter{myEnumCounter}{\value{mySaveCounter}}}
%  }{%
%  \setcounter{mySaveCounter}{\value{myEnumCounter}}\end{list}%
%}
%\newcommand\myEnumReset{\setcounter{mySaveCounter}{0}}

% The old enumerate environment is rewritten, so you need no special command to
% start continuing counting. With the command \myEnumReset you can Reset the couter
% at any place in the text.

% http://www.educat.hu-berlin.de/~voss/lyx/list/enum.phtml

\definecolor{gray}{rgb}{0.4,0.4,0.4}

\begin{document}

%\thispagestyle{empty}
%\pagestyle{plain}

\thispagestyle{fancy}
%\pagenumbering{gobble}
%\fancyhead[location]{text}
% Leave Left and Right Header empty.
\lhead{\textcolor{gray}{\it Limin Ma}}
%\rhead{\textcolor{gray}{\thepage/\totalpages{}}}
%\rhead{\thepage}
\renewcommand{\headrulewidth}{0pt}
\renewcommand{\footrulewidth}{0pt}
\fancyfoot[C]{\footnotesize \textcolor{gray}{}}
%A copy of this curriculum vitae, publications and 
%talk slides are available for download at
%http://www.stanford.edu/$\sim$sundaes/application}}


%\pagestyle{myheadings}
%\markboth{Sundar Iyer}{Sundar Iyer}

\vspace*{0.4cm}
\begin{center}
{\huge \bf Limin Ma}
\vspace*{0.25cm}
\end{center}

\begin{small}

%===================================
\begin{tabbing}
\=xxxxxxxx\=xxxxxxxx\=xxxxxxxx\=\kill
\begin{tabular*}{\linewidth}{l@{\extracolsep{\fill}}r}

Gender: Female& Phone: (814) 852-9588 \\
Address: 509 Oakwood Avenue, State College, PA-16803 &  Email: maliminpku@gmail.com\\
  & Alt: lum777@psu.edu \\
\end{tabular*}
\end{tabbing}

\vspace*{0.2cm}


%==========================================
%\vspace{0.20cm}

\subsection*{PARTICULARS}

%{\color{DarkSeaGreen}} \hrule
\hrule
\vspace{0.2cm}
%%%%%%%%%%%%%%%%%%%%%%%%%%%%%%%

\subsubsection*{EDUCATION}
%\vspace{0.2cm}

%\begin{tabbing}
%xxxxxxxx\=xxxxxxxx\=xxxxxxxx\=xxxxxxxx\=\kill
%\>{\bf Academic record}\\[.7em]

%\>\begin{tabular}{|l|l|l|}
%\hline
%Certificate	&	Place of study	&	Year\\
%\hline
%Ph. D. in Computer Sc.  & Stanford University     & {\it 2000-, Defended 2003} \\
%\hline
%M. S. in Computer Sc. & Stanford University     & {\it June 2000} \\
%\hline
%B. Tech in Computer Sc. and Engg. & I.I.T. Bombay  &	{\it April 1998} \\
%\hline
%\end{tabular}
%\end{tabbing}

%xxxx\=xxxxxxxx\=xxxxxxxx\=xxxxxxxx\=\kill

\begin{tabbing}
xxxx\=xxxxxxxx\=xxxxxxxx\=xxxxxxxx\=\kill
%\>\begin{tabular*}{6.1in}{lr}

\>\begin{tabular*}{0.9\linewidth}{l@{\extracolsep{\fill}}r}
\bf{Pennsylvania State University}& State College, USA     \\
Postdoc in Department of Mathematics  &  {\it Aug.2018 - present}\\
Supervisor: Prof. Jinchao Xu & \\
&\\

\bf{Peking University} & Beijing, China \\
Ph.D. in Department of Scientific and Engineering Computing & {\it Sep.2013 - Jul.2018}\\
Supervisor: Prof. Jun Hu & \\
&\\

\bf{Wuhan University} & Hubei, China \\
B.S. in Department of Computational Mathematics & {\it Sep.2009 - Jul.2013}
\end{tabular*}
\end{tabbing}

\subsubsection*{RESEARCH INTERESTS}
%\hrule
%\vspace{0.2cm}

\begin{list}{}{}
\item My research interest  regards the approximation by finite elements of partial differential equations. In particular, I worked on the following areas: 
\item \begin{itemize}
\item Nonconforming finite element methods for eigenvalue problems
\item Superconvergence of nonconforming elements and mixed elements
\item Finite element methods for linear elasticity problems
\end{itemize}
\end{list}

\subsubsection*{DISSERTATION}

\begin{list}{}{}
\item Title: "High accuracy methods for eigenvalue problems by nonconforming elements"  \\
Advisor: Prof. Jun Hu
\item My thesis concentrates on the analysis of high accuracy methods and proposes some algorithms to improve the accuracy of eigenvalues by finite element methods.  It includes:
\begin{itemize}
\item The first asymptotic expansions of eigenvalues by the nonconforming Crouzeix-Raviart element and the enriched Crouzeix-Raviart element 
\item Design two types of asymptotically exact a posteriori error estimators
\item Propose the penalized Crouzeix-Raviart element which aims to improve the accuracy of large amounts of eigenvalues
\item Prove an optimal superconvergence result for two nonconforming elements
\end{itemize}
\end{list}

 
\subsection*{PUBLICATIONS}
\hrule
\vspace{0.2cm} 
\begin{enumerate}
	\item J. Hu and \textbf{L. Ma}*, ``A Penalized Crouzeix--Raviart Element Method for Second Order Elliptic Eigenvalue Problems",  {\it Journal of Scientific Computing, 74(3):1457-1479}, 2018.
	
	\item J. Hu and \textbf{L. Ma}*, ``Asymptotically Exact A Posteriori Error Estimates of Eigenvalues by the Crouzeix--Raviart Element and Enriched Crouzeix--Raviart Element",  {\it SIAM Journal on Scientific Computing, 42(2): A797--A821}, 2020.
	
	\item J. Hu and \textbf{L. Ma}*, ``Asymptotic Expansions of Eigenvalues by both the Crouzeix-Raviart and Enriched Crouzeix-Raviart elements", {\it Mathematics of Computation}, accepted, 2021.

	
	\item J. Hu, \textbf{L. Ma}* and R. Ma, ``Optimal Superconvergence Analysis for the Crouzeix-Raviart and the Morley elements",  {\it Advances in Computational Mathematics 47(4): 1-25}, 2021.

	\item \textbf{L. Ma}*. ``Superconvergence of Discontinuous Galerkin Methods for the Scaler Elliptic Problems and Linear Elasticity Problems", {\it Journal of Scientific Computing, 88(3):1-20}, 2021

	\item Q. Hong, J. Hu, \textbf{L. Ma}* and J. Xu. ``Extended Galerkin Method for Linear Elasticity with Strongly Symmetric Stress Tensor",  {\it Numerische Mathematik}, accepted, 2021
	
	\item Jonathan Seigel, \textbf{L. Ma} and J. Xu. ``Uniform Approximation Rates and Metric Entropy of Barron Spaces", submitted, 2021
	
	\item \textbf{L. Ma}* and Shudan Tian. ``New Fourth Order Postprocessing Techniques for Plate Buckling Eigenvalues by Morley Element", submitted, 2021% to {\it SIAM Journal on Scientific Computing}.
	
	\item Q. Hong, J. Hu, \textbf{L. Ma} and J. Xu. ``A new approach and efficient preconditioner for time dependent Ginzburg-Landau problems", in preparation
	
	\item Q. Hong, \textbf{L. Ma} and J. Xu. ``Extended Galerkin Method for Stokes Problems", in preparation
    
\end{enumerate}

%\vspace{0.1cm}
\subsection*{TEACHING EXPERIENCE}
\hrule
\vspace{0.2cm}

\begin{itemize}

\item {\bf Instructor.} MATH 230:  Calculus and Vector Analysis, Spring 2020, Pennsylvania State University. 

\item {\bf Instructor.} MATH 251: Ordinary and Partial Differential Equations , Spring 2019, Pennsylvania State University.

\item {\bf Teaching Assistant.} MATH 555: Numerical Optimization, Prof. Jinchao Xu, Spring 2021, Pennsylvania State University.

\item {\bf Teaching Assistant.} MATH 597(section 003): Special Topics, Prof. Jinchao Xu, Spring 2019, Pennsylvania State University.

\item {\bf Teaching Assistant.} MATH 556: Finite Element Methods , Prof. Jinchao Xu, Fall 2018, Pennsylvania State University.

%\item {\bf Teaching Assistant.} An Introduction to Applied Mathematics,  Prof. Jun Hu, Fall 2016, Peking University  
\end{itemize}
\subsection*{ACADEMIC ACTIVITIES}
\hrule
\vspace{0.2cm}
\begin{itemize}
	\item Attend the 2019 AMS-JMM at Baltimore, January 16-19, 2019.
	\item Attend the fall 2018 FE Circus at Delaware, November 9-10, 2018.
	\item Co-organizer of 4th Graduate Forum on Numerical Methods for Partial Differential Equations, Peking University, China, July 2016.
	\item Co-organizer of 2nd Beijing Graduate Forum on Computational Mathematics, Peking University, China, August 2015
\end{itemize}
 
\subsection*{ACADEMIC HONORS}
\hrule
\vspace{0.2cm}
\begin{itemize}
	\item Award for Scientific Research, Peking University, 2017.
	\item Special Scholarship for Scientific Research , Peking University, 2017.
\end{itemize}
 
\subsection*{PRESENTATIONS}
\hrule
\vspace{0.2cm}

\begin{enumerate}
    \item 
       {\it 15th Annual Meeting of China Society for Industrial and Applied Mathematics}, Qingdao, China, October 2017. 

    \item  
       {\it 11th National Conference on Computational Mathematics}, Xi'an, China, July 2017. 

    \item 
       {\it 9th National Conference on Finite Elements}, E'mei, China, August 2016. 

    \item  
       {\it 4th Graduate Forum on Numerical Methods for Partial Differential Equations}, Peking University, China, July 2016. 
   
\end{enumerate}
 
 

%\newpage

\vspace{0.1cm}
%\begin{multicols}{2} [\subsection*{REFERENCES}]

\subsection*{REFERENCES}
\hrule
\vspace{0.2cm}
 
\begin{footnotesize}

\begin{multicols}{2} 
\noindent 
Prof. \textbf{Jinchao Xu} (Postdoc Advisor) \\ 
Dept. of Mathematics \& Penn State University\\
University Park, PA 16802, USA\\
xu@math.psu.edu \\
www.math.psu.edu/xu\\

\columnbreak

\noindent
Prof. \textbf{Jun Hu} (Ph.D. Thesis Advisor)\\ 
School of Mathematical Sciences \& Peking University \\
Beijing, 100871, China \\  
hujun@math.psu.edu.cn \\


\end{multicols}
 %\begin{tabbing}
%xxxx\=xxxxxxxxxxxxxxxxxxxxx\=xxxx\=xxxxxxxxxxxxxxxxxxxxxxxxxxxxxxxxxxxxxxxxx\=\kill
%\>{Nick McKeown}\> :\>nickm@stanford.edu\\
%\>{Balaji Prabhakar}\>:\>balaji@stanford.edu\\
%\>{Ajit Shelat}\>:\>ajit\_shelat@pmc-sierra.com\\
%\>{S.S.S.P. Rao}\>:\>ssspr@cse.iitb.ernet.in\\
%\end{tabbing}
%\newpage

\end{footnotesize}
\end{small}
\end{document}
%%%%%%%%%%%%%%%%%%%%%%%%%%%%%%%%%%%%%%%%%%%%%%%%%%%%%%%%%%%%%%%%%%%%%%%%
